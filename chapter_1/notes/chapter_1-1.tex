
###1.1 什么是操作系统

[TOC]

操作系统(Operating System,简称OS)是管理和控制计算机硬件与软件资源的计算机程序,是直接运行在“裸机”上的最基本的系统软件,任何其他软件都必须在操作系统的支持下才能运行。(来自度娘)

####操作系统在计算机系统中的地位:

|计算机组成部分|面向的研究、适用人群|
|---|---|
|应用软件|应用用户|
|系统工具|应用开发人员|
|操作系统|应用开发人员|
|计算机硬件|操作系统开发人员|

####操作系统目标:

有效性(系统管理人员的观点)
>管理和分配硬件、软件资源,合理地组织计算机的工作流程。

方便性(用户的观点)
>提供良好的、一致的用户接口,弥补硬件系统的类型和数量差别。

可拓展性(开放的观点)
>硬件的类型和规模、操作系统本身的功能和管理策略、多个系统之间的资源共享和互操作。

####OS是计算机资源管理器:

管理对象:
>CPU、存储器、外部设备、信息(数据和软件)

管理内容:
>资源当前状态(数量和使用状态)、资源的分配、回收和访问的操作、相应管理策略(包括用户策略)

####OS是拓展机:

在裸机上添加:设备管理、文件管理、存储管理(针对内存和外存)、处理器管理(针对CPU)

屏蔽异构的计算机硬件,提供统一的抽象接口。
>例如:Windows不像Macos一样预装了很多硬件设备的驱动,但是Windows也因此比较轻巧。

####OS是用户使用系统的接口:

系统命令:
>命令行、菜单式、命令脚本式、图形用户接口GUI

系统调用:
>形式上类似于过程调用,在应用编程中使用。

####操作系统举例:

MS OS
>MS DOS, MS Windows 3.x, Windows 95,$\cdots$,Windows 10,$\cdots$

类UNIX
>BSD, SRV4, Mac OS X,Linux,Android,Minix,$\cdots$
(p.s.:Minix是其中唯一一个微内核的操作系统,其他都为宏内核。)

实时OS
>交换机、工控计算机,例如:VxWorks,pSoS,Nucleus
(对时间的精度要求很高)



