###1.2 操作系统发展历史

[TOC]

####需求推动发展:

器件的发展:
>CPU处理能力、内存容量和速度、$\cdots$
 CPU能耗上无法做到摩尔定律,Intel要活路,走多线程多核的路,~~以此为自己续上几秒~~。(@Loveall Wang非要我改成“续上几年”。) )

提高资源的利用率和系统性能;
>多道系统、分时系统、线程、虚拟化技术。

方便用户:
>文本终端、图形界面、网络服务,$\cdots$

####电子管时代(1945-1955)

工作方式
>用户既是程序员,又是操作员,是计算机专业人员。
>编程语言为机器语言
>输入输出为纸带或卡片

计算机工作特点:
>用户独占全机:不出现资源被其他用户占用,资源利用率低。
>CPU等待用户:计算前,手工装入纸袋或卡片;计算完成后,手工卸去纸带或卡片;CPU利用率低。

####晶体管时代(1955-1965)

利用磁带把若干个作业分类编成作业执行序列,每批作业由一个专门的监督程序(Monitor)自动一次处理。可使用汇编语言开发。

批处理中的作业组成:
>用户处理
>数据
>作业说明书(作业控制语言)

批:
>供一次加载的磁带或磁盘,通常由若干个作业组装成,在处理中使用一组相同的系统软件(系统带)

#####典型FMS作业的结构

不知道是个什么东西。。。

#####脱机批处理操作

由于计算机的更新换代,会出现新机子和旧机子同时可用的情况,且由于两者成本都很高(新的机时被大家排满,旧的不舍得不用),所以会出现了**脱机批处理操作**
操作形式是
1.一般用户(多半是熟悉了原有机型操作流程的工程师)在旧电脑上编好程序(磁带或磁盘),并预约好新机子的使用时间;
2.把磁带或磁盘交给新机子的操作员(专门操作新机子,一般很重要,负责它的维护和使用)
3.操作员在预约好的工时中将新机子调试好,运行磁盘或磁带(就算运行不起来也没办法,这些操作员的时间都很紧的,不容分神)
4.操作员将运行结果交给工程师,工程师根据结果来使用数据或者进行相应的修改。

####集成电路时代(1965-1980)

多道批处理的运行特征:
>多道:内存中同时存放几个作业
>宏观上并行运行:都处于运行状态,但都未运行完
>微观上串行运行:各作业交替使用CPU

多道批处理系统的特点

优点:
>资源利用率高:CPU和内存利用率高
>作业吞吐量大:单位时间内完成的工作总量大

缺点;
>用户交互性差:整个作业完成后或中间出错时,才和用户交互,不利于调试和修改
>作业平均周转时间长:短时间的周转时间显著增强

#####单道与多道程序比较

######内存布局:

单道程序:
|Operating System|
|---|
|user program area|

多道程序:
|Operating System|
|---|
|job 1|
|job 2|
|job 3|
|job 4|
|$\cdots$|

######执行序列:

|Program A|Run_A|Wait|Wait|Wait|Run_A|Wait|Wait|Wait|
|----||||||||
|Program B|Wait|Run_B|Wait|Wait|Wait|Run_B|Wait|Wait|
|Program C|Wait|Wait|Run_C|Wait|Wait|Wait|Run_C|Wait|
|Combined |Run_A|Run_B|Run_C|Wait|Run_A|Run_B|Run_C|Wait|


#####分时系统(time-sharing system)

分时是指多个用户分享使用同一台计算机,多个程序分时共享硬件和软件资源。
>多个用户分时共享:单个用户使用计算机效率低,因而允许多个应用程序同时在内存中,分别服务于不同的用户
>前台和后台程序(foreground & background)分时:后台程序不占用终端输入输出,不与用户交互
>通常按时间片(time slice)分配:各个程序在CPU上执行的轮换时间

例如:MULTICS - Unix的“前辈”

####微处理器时代(1980-)

随着大规模集成电路的发展,芯片在每平方厘米的硅片上可以集成数千个晶体管,于是个人计算机时代到来了。

个人计算机和工作站领域中主流操作系统
>MS-DOS广泛用于IBM PC及其他采用Intel 80X86芯片的计算机
>UNIX在工作站和高档计算机领域(如网络服务器)占据了统治地位,尤其对于采用高性能RISC芯片的计算机

操作系统的特点:
>人机交互性好:在调试和运行程序时由用户自己操作。
>共享主机:多个用户同时使用。
>用户独立性:对每个用户而言好象独占主机。

#####网络/分布式操作系统

Network Operating System(NOS):
>provides mainly file sharing
>Each computer runs independently from other computers on the network.
>(p.s.:every thing you send by third-party system can be viewed by their operators.pay attention to your message security)

Distributed Operating System(DOS):
>gives the impression there is a single operating syetem controlling the network.
>network is mostly transparent - it's a powerful vritual machine.

#####集群系统/网络系统

Clustered Systems
>Clustering allows two or more systems to share external storage and balance CPU load.
>>*Asymmetric clustering.* one server runs the application while other server standby.
>>*Symmetric clustering.* all N hosts are running the application.

Closely coupled system:
>processors also have their own external memory
>communication takes place through high-speed channels
>Provides high reliability.

实例:
>Kerrighed, OpenSSI,openMosix

Grid Computing System
>建立在Internet技术、web技术、高性能计算等技术之上的综合软硬件的基础设施,采用开放标准,为动态参与的多个机构所构成组成的虚拟组织协同完成某类科学、工程或工业上的应用提供可拓展的、安全的、一致的、普及的、高效的大规模资源有效共享
>共享主要不在于文件交换,而在于对计算机、软件、数据和其他资源的直接接入使用,这是工业界、科学界中大量出现的协同解决问题和资源代理策略的需要
>这种共享必须被高度控制,资源提供者和消费者要清晰和详细的定义哪些资源可被共享,谁可享用这些资源,及共享发生的条件。

#####云计算技术(IaaS)

|Servers|$\to$|Provisioning Manager|$\to$|Virtual Machines|
|---|||||
|Storage|$\to$|Provisioning Manager|$\to$|Virtual Machines|

#####虚拟化技术

相当于一个操作系统就是一个程序,由其他的操作系统运行着。可以看作虚拟机的大型版。

原本的机器:
|运行应用|
|---|
|操作系统|
|硬件管理|

运行虚拟化技术的机器:(示意图不会画,画的真难看,见谅。)
||||||
|---|---|---|---|---|
||运行应用_1||运行应用_2||
||操作系统_1||操作系统_2||
||硬件管理_1||硬件管理_2||
||||||
|虚拟机管理器|:可以给各种操作系统一个统一的、|抽象的接口,|使得不同的机器可以起到有同样的运行效果。|(消除底层代码的差异)|
|硬件管理|||||

####MINIX 历史

Tenny Bob等编写了一个在用户看来与UNIX完全兼容,然而内核全新的操作系统
>名字由来: mini UNIX
>1987年发布第一版
>不受任何商业许可证的限制,适用于教学
>内核代码量:4000
>比UNIX晚出现十年,并且其代码采用了一种更加模块化的组织方法
>是LINUX的“前辈”

目前的版本:MINIX 3
>本课程以MINIX 3为例子
